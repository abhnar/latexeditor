\documentclass[hidelinks,12pt,twoside]{iiscthes}
\usepackage[pdftex]{graphicx}
\usepackage{amsmath}
\usepackage{amsfonts}
\usepackage{color}
\usepackage{multirow}
\usepackage{cite}

%#*%commands_

\begin{document}
 

\chapter{FEM Formulation}
%This section is copied from Lee Shin-Hao change it to avoid plagaris
\section{Triangular Elements}
The finite element method approximates an infinite-dimensional function space with a finite-dimensional subspace constructed with the basis functions defined on a finite number of elements. Basis functions employed for a certain application must be consistent with the physical properties of the approximated quantity. 

\begin{figure}[h]
	\includegraphics{./images/triangles_basis.pdf}
	\caption{Basis triangles}
	\label{fig:tri_base}
\end{figure}
The linear nodal basis functions, expressed in terms of the local-area coordinates or the simplex coordinates, are given by,
\begin{equation}
	N_i(L_1, L_2, L3) = L_i, \;i = 1,2,3
\end{equation}
where each of the functions is defined on a vertex of a triangular element as in Fig.\ref{fig:tri_base}.
These basis functions provide a first-order approximation to a scalar field within each element. More widely used than the linear element is the quadratic element, which offers a better convergence of numerical solutions.
\begin{equation}
\begin{split}
	N_1 = (2L_1 -1)L_1 \quad
	N_2 = (2L_2 -1)L_2\quad
	N_3 = (2L_3 -1)L_3\\
	N_4 = 4L_1 L_2\quad\quad\quad\quad
	N_5 = 4L_2 L_3\quad\quad\quad\quad
	N_6 = 4L_3L_1
\end{split}
\end{equation}



\bibliographystyle{ieeetr}
%#*%bib_
\end{document}




